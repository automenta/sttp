
% This is a template page for listing figures

% You can also use pandoc's markdown support for figures and captions (see
% http://www.johnmacfarlane.net/pandoc/README.html#pictures-with-captions).  In
% this case, you don’t need this file. 
%
% But if your defpub includes more complex figures like flowcharts and
% diagrams, we encourage you to use this file.

\newpage

% Example figure made from svg→eps (encapsulated postscript). But this can also
% work with a simple png image file, however please try to use scalable image
% formats.

\vfill
% This helps filling the page for centering the figure vertically on the page

\begin{figure}[h!]
    \centering
    \includegraphics[width=0.95\textwidth]{src/figure.eps}
    \caption{Caption for the figure}
    \label{fig:figure}
    % You can later refer to this figure in the body of your defpub. See
    % https://en.wikibooks.org/wiki/LaTeX/Labels_and_Cross-referencing
\end{figure}

\vfill

% Example of two figures next to each other

\begin{figure}[h!]
    \centering
    \begin{subfigure}[b]{0.45\textwidth}
        \includegraphics[width=\textwidth]{src/left-figure.eps}
        \caption{Caption for the left figure}\label{fig:left-figure}
    \end{subfigure}
    \hfill
    \begin{subfigure}[b]{0.45\textwidth}
        \includegraphics[width=\textwidth]{src/right-figure.eps}
        \caption{Caption for the right figure}\label{fig:right-figure}
    \end{subfigure}

    \caption{Caption for the two figures}
    \label{fig:two-figures}
\end{figure}

\vfill

% Example of flowchart made from TeX, usually with TikZ (see
% http://www.texample.net/tikz/)

\begin{figure}[h!]
    \centering
    
% Define block styles
    \tikzstyle{block} = [rectangle, draw, 
        text width=20em, text centered, minimum height=2em]
    \tikzstyle{decision} = [diamond, draw, fill=red!20, 
        text width=15em, text badly centered, minimum height=2em]
    \tikzstyle{line} = [draw, -latex']

    
\sffamily 
\begin{tikzpicture}[node distance = 2.2cm, auto]
    % Place nodes
    \node [block] (block1) {This is the first step of the flowchart in a block};
    \node [block, below of=block1] (block2) {This is the second step, underneath the first block with an arrow};
    \node [block, below of=block2] (block3) {This is another second step, placed undearneath the second block with an arrow from the first block too};

    % Draw edges
    \path [line] (block1) -- (block2); % This is a straight arrow
    \path [line] (block1) to[out=0,in=0] (block3); % This is a bent arrow
\end{tikzpicture}

    \caption{Caption for the flowchart}
    \label{fig:flowchart}
\end{figure}

\vfill
